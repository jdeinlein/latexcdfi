%   LaTeX Template für den Fachbereich Kunstgeschichte am CDFI
%   Stand Februar 2025
%   Erstellt von Jürgen Deinlein unter Verwendung von DeepSeek R1

%---------------------------------------------------------------------------
%  Dokumentenklasse und verwendete Pakete
%---------------------------------------------------------------------------

\documentclass[12pt,a4paper]{article} % Schriftgröße kann hier geändert werden
\usepackage[utf8]{inputenc} % UTF-8 encoding
\usepackage[top=20mm, left=30mm, bottom=20mm, right=30mm]{geometry} % Seitenränder mit 2 cm oben und unten, 3 cm für die Ränder
\usepackage{times} % Times New Roman font
\usepackage[ngerman]{babel} % Neue Deutsche Rechtschreibung
\usepackage[T1]{fontenc} % Font encoding
\usepackage[style=footnote-dw, isbn=false, url=false, backend=biber]{biblatex} % Bibliography
\usepackage{csquotes} % For proper quotation marks
\MakeOuterQuote{"} % Use proper quotation marks
\usepackage{graphicx} % For including images
\usepackage{caption} % For custom captions
\usepackage{subcaption} % For subfigures
\usepackage[nottoc,numbib]{tocbibind} % Add bibliography to table of contents
\usepackage{fancyhdr} % Custom headers and footers
\usepackage{setspace} % Custom line spacing
\usepackage{ragged2e} % For justifying footnotes

%---------------------------------------------------------------------------
%   Pfade und Kommandos
%---------------------------------------------------------------------------

\graphicspath{ {./images/} } % Pfad zu den Bildern

\newcommand{\dotdot}{ \lbrack\dots\rbrack~} % Shortcut für [...] in Zitaten

\addbibresource{./Vorlage.bib} % Pfad zur Bibliografie Datei !!! WICHTIG !!!!

% Sorgt dafür, dass die Titel kursiv geschrieben sind.
\DeclareFieldFormat{title}{\mkbibemph{#1}}
\DeclareFieldFormat[article,book,inbook,incollection,inproceedings,patent,thesis,unpublished]{title}{\mkbibemph{#1}}

%---------------------------------------------------------------------------
%   Zitate
%---------------------------------------------------------------------------

% Dieses Environment sorgt dafür, dass die Zeilenabstände bei Blockzitaten stimmen
\newenvironment{blockzitat}
{\singlespacing\quote}
{\endquote}

%---------------------------------------------------------------------------
%   Blocksatz für die Fußnoten, schönere Anordnung
%---------------------------------------------------------------------------

\makeatletter
\renewcommand{\@makefntext}[1]{%
	\setlength{\parindent}{0pt} % Remove paragraph indentation
	\setlength{\hangindent}{1.5em} % Add hanging indentation for the footnote text
	\justifying % Justify the footnote text
	\makebox[0pt][r]{\@thefnmark.\hspace{0.5em}}% Place the footnote number in the left margin
	#1% Footnote text
}
\makeatother

%---------------------------------------------------------------------------
%   Dokumenteneinstellung
%---------------------------------------------------------------------------

\begin{document}
	\pagenumbering{gobble} % Keine Seitenzahl für das Deckblatt
		
	%---------------------------------------------------------------------------
	%   Deckblatt
	%---------------------------------------------------------------------------
	
	\title{\LARGE{Titel der Arbeit} \\ \Large\itshape{Der Untertitel der Arbeit}}
	\author{{Mein Name} \\ {Universität Greifswald} \\ {Semester Jahr}}
	\date{Eingereicht am [Datum]}
	
	\maketitle
	\vspace*{\fill}
	\begin{flushleft}
		Art der Arbeit (Hausarbeit, Bachelorsarbeit, etc.)\\
		Fächer, Semester \\
		Caspar-David-Friedrich-Institut \\
		Seminar: Seminar Titel \\
		Betreuer: Name \\
		Modul: Modul Name \\
		Matrikelnummer: 123456 \\
		Email: your.email@university.edu
	\end{flushleft}
	\newpage
	
	%---------------------------------------------------------------------------
	%   Inhaltsverzeichnis
	%---------------------------------------------------------------------------
	
	% Fußzeileneinstellungen, macht vor allem, dass die Numerierung am rechten Rand steht
	\pagestyle{fancy}
	\fancyhf{}
	\fancyfoot[R]{\thepage} 
	\renewcommand{\headrulewidth}{0pt} 
	
	\pagenumbering{roman} % Kleine römische Ziffern für das Inhaltsverzeichnis
	\tableofcontents
	\newpage
	
	%---------------------------------------------------------------------------
	%   Haupttext
	%---------------------------------------------------------------------------
	
	\pagenumbering{arabic} % Arabische Ziffern für den Hauptteil
	\onehalfspacing % 1,5-facher Zeilenabstand
	
	\section{Einleitung}
	% Hier kommt eine Einleitung hin.
	
	Das ist ein toller Text mit einer Fußnote zu Demonstationszwecken.\footnote{Ich bin eine Fußnote und soll zeigen, wie das später mal aussehen wird.}
	
	\section{Sektion 1}
	% Erste Sektion.
	
	Here is an example of a quotation with 1.0 line spacing:
	
	\begin{blockzitat}
		Ich bin ein Blockzitat und bin ein bisschen länger, weshalb man mir eine extra Formatierung zugeteilt hat. So sehe ich dann im Fließtext aus, ich sollte eingerückt und mit 1-fachen Zeilenabstand sein. Falls ich das nicht bin, ist irgendwo etwas schief gegangen.
	\end{blockzitat}
	
	\section{Noch eine Sektion}
	% Inhalt dieser Sektion.
	
	%---------------------------------------------------------------------------
	%   Literaturverzeichnis
	%---------------------------------------------------------------------------
	
	\newpage
	\printbibliography[heading=bibintoc,keyword={source},title={Quellenverzeichnis}] % Wenn es Quellen gibt, die nicht in das Literaturverzeichnis sollen, dann kommen sie hier hin.
	\printbibliography[heading=bibintoc,notkeyword={source},title={Literaturverzeichnis}] % Das ist das eigentliche Literaturverzeichnis
	\listoffigures % Abbildunsverzeichnis
	
	%---------------------------------------------------------------------------
	%   Lizensen (Optional)
	%---------------------------------------------------------------------------
	
	\section*{Lizeninformationen}
	CC BY-SA 3.0, einzusehen unter \url{https://creativecommons.org/licenses/by-sa/3.0/} \\
	CC BY-SA 4.0, einzusehen unter \url{https://creativecommons.org/licenses/by-sa/4.0/}
	
	%---------------------------------------------------------------------------
	%   Anhang
	%---------------------------------------------------------------------------
	
	\newpage
	\section*{Anhang}
	% Hier kommen alle Bilder etc. rein.
	\section*{Erklärung}
	Hiermit erkläre ich, dass ich die vorgelegte Arbeit eigenständig verfasst und keine anderen als die im Literaturverzeichnis angegebenen Quellen, Darstellungen und Hilfsmittel benutzt habe. Dies trifft insbesondere auch auf Quellen aus dem Internet zu. Alle Textstellen, die wortwörtlich oder sinngemäß anderen Werken oder sonstigen Quellen entnommen sind, habe ich in jedem einzelnen Fall unter genauer Angabe der jeweiligen Quelle, auch der Sekundärliteratur, als Entlehnung gekennzeichnet. Die Nutzung von chatgpt und/oder anderen KI-Anwendungen ist kenntlich gemacht. \\ \\
	Ich erkläre hiermit weiterhin, dass die vorgelegte Arbeit zuvor weder von mir noch -- soweit mir bekannt ist -- von einer anderen Person an dieser oder einer anderen Hochschule eingereicht wurde. \\ \\
	Darüber hinaus ist mir bekannt, dass die Unrichtigkeit dieser Erklärung eine Benotung der Arbeit mit der Note "nicht ausreichend" zur Folge hat und dass Verletzungen des Urheber- rechts strafrechtlich verfolgt werden können.
\end{document}